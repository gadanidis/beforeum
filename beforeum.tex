\documentclass{article}

\title{Before the rise of um}
\author{Derek Denis and Timothy Gadanidis}

% SIL font
\usepackage{fontspec}
\setmainfont{Charis SIL}

% margins
\usepackage[margin=1in]{geometry}

% bibliography
\RequirePackage[
    backend=biber,
    style=apa,
    sortcites=true,
sorting=nyt]
{biblatex}
\RequirePackage[american]{babel}
\bibliography{beforeum.bib}
\RequirePackage{csquotes}
\RequirePackage[american]{babel}
\DeclareLanguageMapping{american}{american-apa}

% colon postnotes
\renewcommand{\postnotedelim}{%
  \iffieldnums{postnote}
    {\addcolon~}
    {}}
\DeclareFieldFormat{postnote}{#1}
\DeclareFieldFormat{multipostnote}{#1}

% table
\usepackage{booktabs}

% figures
\usepackage{graphicx}

\begin{document}

\maketitle

\section{Introduction}

% TODO: introduction should reflect chapter as a whole

One of the most dramatic discourse-pragmatic changes in twentieth-century
English has progressed under the radar of laypeople and (until recently)
linguists: the rise of \emph{um} as the predominant variant of the `filled
pause' variable (UHM) at the expense of \emph{uh} \parencite{tottie2011,
fruehwald2016, wielingetal2016}.
\textcite[43]{fruehwald2016} documents this ``textbook'' change over 100+ years
of apparent time:
\emph{um} increases incrementally between generations and the rise is led by
women.
In this chapter, we investigate (UHM) at an early stage of change to determine
what triggered the rise of \emph{um}.

\section{Change in progress}

The rise of \emph{um} has now been described extensively in the variationist and
corpus-linguistic literature, across a number of corpora and speech communities.

% TODO: acton2011

In the British National Corpus, \textcite{tottie2011} observed that \emph{um}
was used more frequently than \emph{uh} by women, younger speakers, and more
educated speakers; men, older speakers and educated speakers used (UHM) more
often overall.
\textcite{fruehwald2016}

% TODO: wielingetal2016

While these accounts demonstrate definitively that a change is underway, an
explanation for the change remains elusive.
What was the trigger for this ``textbook'' change?

% TODO: explain functional expansion hypothesis

In this chapter, we investigate data from before the rise of \emph{um} with the
goal of evaluating the functional expansion hypothesis.

\section{Data}

The data for this study are from the \emph{Farm Work and Farm Life Since 1890}
oral history collection \parencite{denis2016}.
The corpus consists of oral history interviews with 155 elderly farmers,
recorded in 1984.
The corpus covers five regions of Ontario, Canada: Temiskaming, Essex, Dufferin,
Niagara Region, and Eastern Ontario; for this study, speakers from the latter
two regions were considered.
Speaker birth years range from 1891 to 1919, just before \emph{um} began to take
off per \textcite{fruehwald2016}.

We extracted each instance of \emph{uh} and \emph{um} from the transcripts,
excluding unrelated instances such as \emph{uh-oh}.
Tokens from the two much-younger interviewers was also extracted, and analyzed
separately.
The transcription protocol emphasized faithful reproduction of \emph{uh} and
\emph{um}.

\section{Coding}

We coded for the following social factors:
year of birth, gender, and region (Niagara or Eastern Ontario).

To operationalize the functional expansion hypothesis, we coded for utterance
position (initial or non-initial).

% TODO: cliticization

\section{Results}

Table~\ref{t:comparison} shows how our data compare with previous communities
analyzed.
The first block summarizes our data from Niagara and Eastern Ontario, as well as
F-INT and M-INT, the two younger interviewers.
The second block summarizes results from previous work on the Switchboard corpus
\parencite{switchboard}, the Fisher corpus \parencite{fisher}, the Philadelphia
Neighborhood Corpus (PNC) \parencite{labovrosenfelder2011}, and the British
National Corpus (BNC) \parencite{bnc}.
The numbers for all of these other corpora are drawn from
\textcite{wielingetal2016}.

\begin{table}[ht!]
    \centering
    \begin{tabular}{lrrrrrr}
        \toprule
                  & Raw N       & Raw N       & \%          & Mean              & Mean             & Mean		\\
        Community & \textit{uh} & \textit{um} & \textit{um} & \textit{uh} /1000 & \textit{um}/1000 & UHM/1000	\\
        \midrule
        Niagara   & 1864        & 357         & 16.1        & 21.3              & 4.1              & 25.4	\\
        E. Ont.   & 1563        & 168         & 9.7         & 22.6              & 2.4              & 25.0		\\
        F-INT     & 321         & 318         & 49.8        & 12.4              & 12.3             & 24.7		\\
        M-INT     & 255         & 51          & 16.7        & 13.2              & 2.6              & 15.8		\\
        \midrule
        Switchboard  & ---         & ---         & 28.3        & 22.1              & 7.5              & 29.6		\\
        Fisher    & ---         & ---         & 64.1        & 6.8               & 9.9              & 16.7		\\
        PNC       & ---         & ---         & 27.6        & 13.2              & 4.5              & 17.7		\\
        BNC       & ---         & ---         & 46.1        & 4.5               & 4.3              & 8.8			\\
        \bottomrule
    \end{tabular}
    \caption{Cross-community comparison}
    \label{t:comparison}
\end{table}

As can be seen in the table, \emph{um} is less frequent in our data compared to
the more recent corpora; the female interviewer uses it around half the time,
while the male interviewer's rate is comparable to the farmers'.
Relative frequency of (UHM) taken as a whole is on par with other corpora, but
we are cautious about making such a comparison because each corpus was collected
and transcribed differently \parencite[for related discussion,
see][]{pichler2010}.

\printbibliography

\end{document}
