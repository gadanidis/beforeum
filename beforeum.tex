\documentclass[11pt]{article}

\title{Before the rise of um}
\author{Derek Denis and Timothy Gadanidis}

% SIL font
\usepackage{fontspec}
\setmainfont{Charis SIL}

% margins
\usepackage[margin=1in]{geometry}

% bibliography
\RequirePackage[
    backend=biber,
    style=apa,
    sortcites=true,
sorting=nyt]
{biblatex}
\RequirePackage[american]{babel}
\bibliography{beforeum.bib}
\RequirePackage{csquotes}
\RequirePackage[american]{babel}
\DeclareLanguageMapping{american}{american-apa}

% colon postnotes
\renewcommand{\postnotedelim}{%
  \iffieldnums{postnote}
    {\addcolon~}
    {}}
\DeclareFieldFormat{postnote}{#1}
\DeclareFieldFormat{multipostnote}{#1}

% table
\usepackage{booktabs}

% figures
\usepackage{graphicx}

% extracts
\usepackage{newfloat}
\usepackage{caption}
\usepackage{mdframed}

\DeclareFloatingEnvironment[name=Extract]{extract}
\usepackage{dialogue}

\begin{document}

\maketitle

\section{Introduction}

One of the most dramatic discourse-pragmatic changes in twentieth-century
English has progressed under the radar of laypeople and (until recently)
linguists: the rise of \emph{um} as the predominant variant of the `filled
pause' variable (UHM) at the expense of \emph{uh} \parencite{tottie2011,
fruehwald2016, wielingetal2016}.
\textcite[43]{fruehwald2016} documents this ``textbook'' change over 100+ years
of apparent time:
\emph{um} increases incrementally between generations and the rise is led by
women.
Why \emph{um}? Why did this change occur?
In this chapter, we investigate (UHM) at an early stage of change to determine
what triggered the rise of \emph{um}.
We take up the hypothesis that the rise of \emph{um} was connected to the
development of a new discourse function for the variable (UHM), that \emph{um}
came to be favoured with.
We remain agnostic about what the function is, but follow \textcite{tottie2016}
and \textcite{fruehwald2016} who suggest a correlation between utterance
position and function; and specifically \textcite[46]{fruehwald2016} who suggests
that ``turn-initial \emph{um} may be the best candidate for a new
discourse function coming into use.''
We follow essentially a variationist approach, first treating \emph{um} and
\emph{uh} as variants of a linguistic variable and using proportional analysis
to assess the role of social and linguistic factors.
We augment these results with a different quantitative perspective and examine
the relative frequency of the variable itself in discourse to help in our
interpretation.

\section{(UHM) as a pragmatic marker}

For the purposes of this study, we follow \textcite{fruehwald2016},
\textcite{wielingetal2016}, and \textcite{tottie2016}, among others, in treating
\emph{uh} [əː] and \emph{um} [əːm] (also written as \emph{er} and \emph{erm}) as
variants of one variable, termed (UHM).
It should be noted that this is not the only way that the variable context could
be defined.
For instance, \textcite{tottie2018} includes (UHM) as one element of a set
including \emph{well}, \emph{you know}, and \emph{like}, on the basis that all
of the elements are used to indicate speech planning.
However, we argue that treating \emph{um} and \emph{uh} as an individual
variable captures the two words' intuitive and structural similarity\footnote{%
    This is an (in our view justified) extension of the notion of ``structural
    equivalence'' \parencite{pichler2010, tagliamontedenis2010} to phonological/orthographic structure.
}, both variants being phonologically and orthographically identical, modulo
the coda.
Both variants are also single-word constructions which, unlike \emph{well},
\emph{you know}, and \emph{like}, do not appear to be derived from bleached
lexical items, but from apparently non-lexical speech sounds.
As \textcite{fruehwald2016} notes, they have also traditionally been treated as
a unique phenomenon in the psycholinguistic literature.

The exact nature of (UHM) as a linguistic feature is not a trivial question.
A great deal of ink has been spilled over whether they are produced consciously
or unconsciously, and what their purpose is.
For example, \textcite[41--42]{maclayosgood1959} characterize (UHM) as a
floor-management device which speakers insert to indicate that they do not want
to be interrupted when hesitating over what to say.
\textcite{levelt1983, levelt1989} describes (UHM) as an involuntary noise
produced as a result of production problems: ``\emph{er} apparently signals that
at the moment when trouble is detected, the source of the trouble is still
actual or quite recent. But otherwise, \emph{er} doesn't seem to mean anything.
It is a symptom, not a sign'' \parencite[484]{levelt1989}.

One problem with the involuntary ``symptom'' view is that, as
\textcite{clarkfoxtree2002} point out, speakers have some control over whether
or not they produce (UHM)---%
for example, it can be suppressed in a public speaking context (and indeed
speakers are often counselled to do so).
They argue that (UHM) is an ``interjection'' used to signify a delay, with
\emph{um} signalling longer delays than \emph{uh}.

Recently, \textcite{tottie2016} has put forward the argument that
(UHM) is a pragmatic marker that, in speech, indicates planning.
This is on the basis that (UHM) is used more frequently in contexts requiring
more speaker planning, such as narratives and responses to questions.
\textcite{tottie2017} describes (UHM) as being on a ``cline of lexicalization'',
where apparently-cliticized forms like \emph{and-uh} and \emph{but-uh} are not
perceived as words, but \emph{uh} and \emph{um} alone are.
\textcite[21--22]{tottie2017} hypothesizes that the use of (UHM) between words
and silent pauses, rather than in \emph{and-uh} and its ilk, leads to the
perception of (UHM) as a word in the lexicon, leading it to be available for
conscious use.
If this process was a factor in (UHM)'s diachronic development, we might expect
to see an effect of cliticization
% diachronically --- could be seen as part of gramamticalization/lexicalization
% process

% TODO: Is there something interesting to say about that?

As we note above, the rise of \emph{um} has now been described extensively in
the variationist and corpus-linguistic literature, across a number of corpora
and speech communities.
The typical finding is that women have a higher \emph{um}--\emph{uh} ratio than
men, and that younger speakers have a higher \emph{um}--\emph{uh} ratio than
older ones.
This pattern has been demonstrated in various speech communities and contexts in
the United States \parencite{acton2011, fruehwald2016, wielingetal2016,
lasernaetal2014}, as well as in England and Scotland \parencite{tottie2011,
wielingetal2016}, both in real and apparent time.
\textcite{wielingetal2016} also show that this pattern extends beyond English to
five other Germanic languages: Dutch, German, Norwegian, Danish, and Faroese.

While these accounts demonstrate definitively that a change is underway, an
explanation for the change remains elusive.
What was the trigger for this ``textbook'' change?
\textcite{fruehwald2016} and \textcite{wielingetal2016} both suggest that a new
meaning or function for \emph{um} may have emerged in English\footnote{%
    For \textcite{wielingetal2016}, this is a possible explanation for the
    crosslinguistic nature of the change: a function could have emerged in
    English and then spread through contact to the other Germanic languages.
}.
% TODO: good way of explaining why we still tested this hypothesis blabla
In this chapter, we investigate data from before the rise of \emph{um} with the
goal of evaluating the functional expansion hypothesis.

\section{Data and coding}

The data for this study are from the \emph{Farm Work and Farm Life Since 1890}
oral history collection \parencite{denis2016}.
The corpus consists of oral history interviews with 155 elderly farmers,
recorded in 1984.
The corpus covers five regions of Ontario, Canada: Temiskaming, Essex, Dufferin,
Niagara Region, and Eastern Ontario; for this study, speakers from the latter
two regions were considered.
Speaker birth years range from 1891 to 1919, just before \emph{um} began to take
off per \textcite{fruehwald2016}.

We extracted each instance of \emph{uh} and \emph{um} from the transcripts,
excluding unrelated instances such as \emph{uh-oh}.
Tokens from the two much-younger interviewers was also extracted, and analyzed
separately.
The transcription protocol emphasized faithful reproduction of \emph{uh} and
\emph{um}.

The data were coded for the following social factors:
year of birth, gender, and region (Niagara or Eastern Ontario).
Year of birth and gender were used to operationalize the change-in-progress
hypothesis.
Table % TODO \ref{tab:speakers}
presents a table of speakers by gender, region, and year of birth.

We also coded for two linguistic factors.
To operationalize the functional expansion hypothesis, we coded for utterance
position (initial or non-initial).
(UHM) was defined as ``initial'' if it was the first element in an utterance,
except in the case of \emph{and-} or \emph{but-} cliticization, where (UHM) was
classed as ``initial'' if the containing utterance began with \emph{and-uh} or
\emph{but-uh}.
To test for a potential effect of cliticization (per \citeauthor{tottie2017}'s
\citeyear{tottie2017} suggestion that this may have played a role in (UHM)'s
lexicalization) we coded each token as ``clitic'' if it occurred immediately
following \emph{and} or \emph{but}, and as ``non-clitic'' otherwise.

\section{Results}

\subsection{Proportional frequency}

Table~\ref{t:comparison} shows how our data compare with previous communities
analyzed.
The first block summarizes our data from Niagara and Eastern Ontario, as well as
F-INT and M-INT, the two younger interviewers.
The second block summarizes results from previous work on the Switchboard corpus
\parencite{switchboard}, the Fisher corpus \parencite{fisher}, the Philadelphia
Neighborhood Corpus (PNC) \parencite{labovrosenfelder2011}, and the British
National Corpus (BNC) (\citeyear{bnc}).
The numbers for all of these other corpora are drawn from
\textcite{wielingetal2016}.

\begin{table}[ht!]
    \centering
    \begin{tabular}{lrrrrrr}
        \toprule
                    & Raw N       & Raw N       & \%          & Mean              & Mean             & Mean     \\
        Community   & \textit{uh} & \textit{um} & \textit{um} & \textit{uh} /1000 & \textit{um}/1000 & UHM/1000 \\
        \midrule
        Niagara     & 1864        & 357         & 16.1        & 21.3              & 4.1              & 25.4     \\
        E. Ont.     & 1563        & 168         & 9.7         & 22.6              & 2.4              & 25.0     \\
        F-INT       & 321         & 318         & 49.8        & 12.4              & 12.3             & 24.7     \\
        M-INT       & 255         & 51          & 16.7        & 13.2              & 2.6              & 15.8     \\
        \midrule
        Switchboard & ---         & ---         & 28.3        & 22.1              & 7.5              & 29.6     \\
        Fisher      & ---         & ---         & 64.1        & 6.8               & 9.9              & 16.7     \\
        PNC         & ---         & ---         & 27.6        & 13.2              & 4.5              & 17.7     \\
        BNC         & ---         & ---         & 46.1        & 4.5               & 4.3              & 8.8      \\
        \bottomrule
    \end{tabular}
    \caption{Cross-community comparison}
    \label{t:comparison}
\end{table}

As can be seen in the table, \emph{um} is less frequent in our data compared to
the more recent corpora; the female interviewer uses it around half the time,
while the male interviewer's rate is comparable to the farmers'.
Relative frequency of (UHM) taken as a whole is on par with other corpora, but
we are cautious about making such a comparison because each corpus was collected
and transcribed differently \parencite[for related discussion,
see][]{pichler2010}.

Looking at individual speakers' rates, we can see that all speakers use both
\emph{uh} and \emph{um}, but there is no clear pattern by age
(Figure~\ref{fig:indivage}) or gender (Figure~\ref{fig:indivgender}).

% TODO: these figures probably need to be remade in black and white
%       plus there are a lot of visual improvements that could be made

\begin{figure}[htpb]
    \centering
    \includegraphics[width=0.8\linewidth]{figures/indivage.png}
    \caption{Proportion \emph{um} per speaker by age.}
    \label{fig:indivage}
\end{figure}

\begin{figure}[htpb]
    \centering
    \includegraphics[width=0.8\linewidth]{figures/indivgender.png}
    \caption{Proportion \emph{um} per speaker by gender.}
    \label{fig:indivgender}
\end{figure}

Figure~\ref{fig:apparenttime} shows the proportion of \emph{um} in apparent
time.
In the plot to the left, year of birth is binned into five-year increments,
whereas in the plot to the right, year of birth is continuous.
In both cases, there is a modest trend upward over time.
To determine the possible predictors underlying this trend, in the following
figures we split the data by gender, position and cliticization.
Figure~\ref{fig:apparentgender} shows the pattern when splitting speakers by
gender.
Starting around 1905, women use \emph{um} slightly less often than men do, with
both genders' \emph{um} rates trending slightly upward over time.
Figure~\ref{fig:apparentposition} shows the pattern when splitting tokens by
position (initial vs.\ non-initial).
Starting around 1905, \emph{um} is used more frequently in initial position than
in non-initial position.
Figure~\ref{fig:apparentclitic} shows the pattern when splitting tokens by
cliticization with \emph{and} or \emph{but} and position.
\emph{Um}'s proportional increase appears to be limited to non-cliticized
initial tokens.

\begin{figure}[htpb]
    \centering
    \includegraphics[width=0.8\linewidth]{figures/apparenttime.png}
    \caption{Proportion \emph{um} in apparent time.}
    \label{fig:apparenttime}
\end{figure}

\begin{figure}[htpb]
    \centering
    \includegraphics[width=0.8\linewidth]{figures/apparentgender.png}
    \caption{Proportion \emph{um} in apparent time, by gender.}
    \label{fig:apparentgender}
\end{figure}

\begin{figure}[htpb]
    \centering
    \includegraphics[width=0.8\linewidth]{figures/apparentposition.png}
    \caption{Proportion \emph{um} in apparent time, by position.}
    \label{fig:apparentposition}
\end{figure}

\begin{figure}[htpb]
    \centering
    \includegraphics[width=0.8\linewidth]{figures/apparentclitic.png}
    \caption{Proportion \emph{um} in apparent time, by position and
    cliticization.}
    \label{fig:apparentclitic}
\end{figure}

Figure~\ref{fig:farmertree} shows a conditional inference tree for all farmers.
The model confirms several of the patterns indicated in
Figures~\ref{fig:apparenttime}--\ref{fig:apparentclitic}.
The tree splits first at cliticization, with cliticized (UHM) having the lowest
overall \emph{um} rate.
Within the cliticized tokens, there is a slight difference between noninitial
and initial (UHM), with initial tokens having a higher \emph{um} rate than
noninitial ones.
Looking at the noncliticized side of the tree, we see another positional split,
again with noninitial tokens having a lower rate than initial ones.
Within the initial tokens, there is also an effect of year of birth:
speakers born after 1916 have a much higher \emph{um} rate in noncliticized
initial tokens than those born in 1916 or earlier.
It should be noted % FIXME

\begin{figure}[htpb]
    \centering
    \includegraphics[width=0.8\linewidth]{figures/ctreefarmers.png}
    \caption{Conditional inference tree for farmers.}
    \label{fig:farmertree}
\end{figure}

Figure~\ref{fig:interviewertree} shows a conditional inference tree for the two
interviewers.

\begin{figure}[htpb]
    \centering
    \includegraphics[width=0.8\linewidth]{figures/ctreeinterviewers.png}
    \caption{Conditional inference tree for interviewers.}
    \label{fig:interviewertree}
\end{figure}

Taken together, the results presented in this section appear to show the
beginning of the change toward \emph{um} that has been observed by other
researchers.
While other work has shown that women lead this change, in our data, older women
actually use more \emph{um} than the younger women.

Looking at internal factors, we can see that cliticized forms, like
\emph{and-uh}, favour \emph{uh}.
There is some evidence for positional divergence, possibly consistent with a new
utterance-initial discourse function that favours \emph{um}
\parencite[cf.][, who found no turn-positional difference]{fruehwald2016}.
Conditional inference trees confirm that the internal constraints persist with
the younger speakers, while their baseline \emph{um} rate is higher.

\subsection{Relative frequency}

\textcite{fruehwald2016} tests the hypothesis that functional expansion
triggered the rise of um by considering changes to the relative frequency of
variants over time (e.g., frequency of \emph{um} or \emph{uh} per 10~000 words).
When a new discourse-pragmatic function emerges, we expect that these functions
would add to the relative frequency of the feature, and if the new function is
restricted to one variant, the relative frequency of that variant should rise,
with little change to the relative frequency of the other variant.
In other words, we expect a fishtail pattern as with \emph{computer} and
\emph{typewriter} over time: once \emph{computer} gained its contemporary
meaning, its relative frequency took off as that meaning became more frequent.
This is illustrated in Figure~\ref{fig:fishtail} \parencite[Figure 3
from][]{fruehwald2016}: looking at the proportion of \emph{computer} over
\emph{typewriter} (left graph), \emph{computer} appears to replace
\emph{typewriter} over time; but looking at the relative frequency of each word
(right graph), it's clear that \emph{typewriter} remained stable as
\emph{computer} took off.

\begin{figure}[htpb]
    \centering
    \includegraphics[width=0.8\linewidth]{figures/fishtail.png}
    \caption{Proportional frequency and relative frequency of \emph{computer}
    and \emph{typewriter} \parencite[Figure 3 from][]{fruehwald2016}.}
    \label{fig:fishtail}
\end{figure}

If a new discourse function is what led to the rise of \emph{um}, we should
expect to see a similar fishtail pattern, with \emph{um} rising and \emph{uh}
remaining stable.
Conversely, if \emph{um} were straightforwardly replacing \emph{uh}, we should
expect \emph{uh} to fall concurrently with \emph{um}'s rise.

Figure~\ref{fig:relfreq} shows the frequency of \emph{um} and \emph{uh} per 1000
words for each of farmers.
There is some evidence of a fishtail pattern, but in the opposite direction as
expected: \emph{uh} is increasing as \emph{um} remains relatively stable.
The pattern is more extreme when we split the data by position, as in
Figure~\ref{fig:relfreqposition}.
In initial position, both \emph{um} and \emph{uh} are largely stable, whereas in
noninitial position, \emph{uh} alone is increasing.
Splitting the data again by gender, we can see that the increase can be
attributed to the female speakers---%
there is no apparent increase over apparent time for male speakers,
but the older female speakers have a relatively lower \emph{uh} rate, rising to
match the male speakers by the 1910s.

\begin{figure}[htpb]
    \centering
    \includegraphics[width=0.8\linewidth]{figures/relfreq.png}
    \caption{Name}%
    \label{fig:relfreq}
\end{figure}

\begin{figure}[htpb]
    \centering
    \includegraphics[width=0.8\linewidth]{figures/relfreqposition.png}
    \caption{Name}%
    \label{fig:relfreqposition}
\end{figure}

\begin{figure}[htpb]
    \centering
    \includegraphics[width=0.8\linewidth]{figures/relfreqgenderposition.png}
    \caption{Name}%
    \label{fig:relfreqgenderposition}
\end{figure}

\section{Discussion}

Exploring data from before the rise of \emph{um} has failed to yield insight on
its dramatic rise.
Looking at the proportional frequency, we do find that for a few of the younger
farmers, % TODO: check how many
\emph{um} appears more frequent than \emph{uh} in initial position, and the same
is the case for the two (much younger) interviewers.
Alone, this could be taken as suggestive (but not definitive, given the low
number of speakers involved) evidence in favour of a new, initial-position
function for \emph{um}.

Looking at the relative frequency, however, we see that the pattern is not driven
by an increase of \emph{um} in initial position---%
like \textcite{fruehwald2016}, we do not find strong evidence that a new,
utterance-initial function for \emph{um} is behind the dramatic rise of
\emph{um}.
Instead, we find evidence of a different change: an increase in \emph{uh} in
non-initial position.

See Extracts~\ref{ext:highuhm} and~\ref{ext:lowuhm} for examples of speakers
who use (UHM) frequently and infrequently, respectively.

\begin{extract}
    \begin{mdframed}[leftmargin=10pt,rightmargin=10pt]
        \begin{dialogue}

            \speak{INT} And what types of fruit (.) did you grow?

            \speak{NO-11} Well the \textbf{uh} (.) originally \textbf{uh} when they
            came- \textbf{uh} grandfather bought the property in nineteen hundred
            and \textbf{uh} (.) \textbf{um} (.) to begin with there was very- there
            were very few fruit trees on it and they planted (.) \textbf{uh} (.) our
            orchard of \textbf{uh} (.)  peaches. And \textbf{uh} waiting- while they
            waited for the peaches to come into bearing, they planted raspberries
            between the rows, so it started out as principally a raspberry farm I
            suppose but (.) it evolved into a farm that \textbf{uh} principally grew
            peaches and cherries, mainly sweet~cherries.
        \end{dialogue}
    \end{mdframed}
    \caption{High (UHM) user}\label{ext:highuhm}
\end{extract}

\begin{extract}
    \begin{mdframed}[leftmargin=10pt,rightmargin=10pt]
        \begin{dialogue}

            \speak{INT} Okay. And how much (.) older was the very oldest?

            \speak{NO-36} The oldest was born (\ldots) in eighteen ninety two
            (\ldots) and then my sister Lianne, eighteen ninety four (\ldots)
            Greg, eighteen ninety eight (.) Sally nineteen hundred and one
            (\ldots) I was born nineteen hundred and three (.) and that's it.

            \speak{INT} Okay, and how old was your dad when you were born? At-

            \speak{NO-36} (\ldots) I- (\ldots) how old was my dad when I was born? Oh.

            \speak{INT} I think we had figured out that he was probably
            somewhere around forty five.

            \speak{NO-36} Oh yes.

            \speak{INT} And your mom was?

            \speak{NO-36} Thirty (.) five?

            \speak{INT} Thirty- oh-

            \speak{NO-36} Is that it?

            \speak{INT} Yup. Good.

        \end{dialogue}
    \end{mdframed}
    \caption{Low (UHM) user}\label{ext:lowuhm}
\end{extract}

% TODO: evidence of a different change — unfilled to filled pauses

% TODO: importance of multiple methods — conflicting views on expansion
% quantitative innovation (thanks)

\printbibliography

\end{document}
